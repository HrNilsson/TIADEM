%!TEX root = Main.tex
\documentclass[Main]{subfiles}

\begin{document}

\section{Conclusion} % (fold)
\label{sec:conclusion}
	In this project an implementation of the DFTSP on TelosB motes has been developed. 
	The implementation has been tested on a small scale test setup with two motes in a temperature stable and unstable environment. 
	The test results shows the correlation between synchronization error, beacon period, temperature and skew.
	From the results, it is clear to see that motes exposed to unstable environments need lower synchronization interval to maintain the same synchronization accuracy.

	By using the DFTSP instead of the FTSP a variable synchronization period is introduced. This reduces the overall transmission overhead since the period can be increased when the environment is stable and reduced when unstable.
	In a WSN network this means lower power consumption in all motes.

% section conclusion (end)

\end{document}