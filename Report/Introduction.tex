%!TEX root = Main.tex
\documentclass[Main]{subfiles}

\begin{document}

\section*{Introduction} % (fold)
\label{sec:introduction}

	The ability to have an accurate time awareness in wireless sensor networks is an important functionality, as evident in the research published over the past few decades\cite{FTSParticle,dynamicFTSParticle,RBSarticle,TPSNarticle}.
	\\Time synchronization is important for sensor networks for several reasons.
	Applications such as localization systems demand accurate time information in order to work properly.
	Various forms of medium access control (MAC) protocols, like the TDMA\footnote{Time Division Multiple Access}, use timing information to determine when a sensor can access the wireless channel.
	Lastly time synchronization is also a vital element in duty cycling schemes, which are used to converse power.

	Having the nodes of a network continuously synchronized is further complicated by the phenomenon known as clock drift. 
	Time is measured in ticks of the local oscillator on the nodes, also referred to as the clock.
	Since the exact frequency of the clock oscillation differs from node to node, and is influenced by external factors such as temperature, a periodic time synchronization protocol is needed to maintain an accurate global time estimation.

	This report documents a mini-project made in the \emph{Advanced Embedded Sensor Networks} course at Aarhus University, Department of Engineering. 
	It describes the implementation of the DFTSP (Dynamic Flooding Time Synchronization Protocol), and is based on the work presented in \cite{dynamicFTSParticle}. 
	The DFTSP extends the well-known FTSP (Flooding Time Synchronization Protocol)\cite{FTSParticle} with a dynamic synchronization period. 

% section introduction (end)


\end{document}
