%!TEX root = Main.tex
\documentclass[Main]{subfiles}

\begin{document}

\section{Motivation} % (fold)
\label{sec:motivation}

	The motivation for this mini-project has been to implement and test the DFTSP, since the results from \cite{dynamicFTSParticle} promise improvements of the widely used FTSP. 
	The reader is assumed to have knowledge of the FTSP, which can be found in \cite{FTSParticle}. 

	The improvements are mainly concerned with the trade off between protocol overhead and time estimation error, which is controlled by the synchronization period.
	This period determines the duration of time between synchronization messages, where the nodes have to keep themselves synchronized based on the clock skew estimate from the regression table.
	\\A higher synchronization period is more susceptible to changes in clock skew (clock drift) caused by temperature changes, since the skew estimation has to hold for a longer time period. 
	Errors in the skew estimation and the assumption that the skew is constant within a synchronization period will result in a greater accumulated offset error with a large period than with a small period. 
	This accumulated error should always be kept below the application error bound in order to ensure the desired functionality. 
	\\However, since the synchronization period also is linearly proportional to the energy and bandwidth consumption caused by protocol overhead, a design trade off is present.
	The optimal period would keep the estimation error below the error bound, whilst minimizing the energy and bandwidth consumed by protocol overhead.
	However an optimal period cannot be realized with the FTSP as the factors that affect the accumulated error are time varying! 
	Furthermore a sensor network is also likely to experience spatial differences in temperature, which would make different synchronization periods suitable to different regions in the network.

	Hence the FTSP could be improved by making the synchronization period adapt to the real time conditions of the network. 
	This is precisely the principle being used in the DFTSP. 
	\\The nodes estimate the clock drift of their children by overhearing their synchronization messages.
	A suitable synchronization period for the children is then determined based on the worst estimated clock drift and the application error bound.
	\\Thereby DFTSP can achieve a smaller protocol overhead on average whilst still maintaining the timing error below the error bound.  


% section motivation (end)


\end{document}
